\section{TPM}
\begin{itemize}
  \item Description of the TPM - what it is, and what it's used for
  \item Examples of TPM commands with applied pi-calculus
\end{itemize}

The Trusted Platform Module (TPM) is a specialised chip that stores RSA encryption keys specific for the host system for hardware authentication. It is used as a component on an endpoint device and is used for the Windows BitLocker.
The TPM contains an RSA key pair called the Endorsement Key (EK), together with an owner-specified password. The Storage Root Key (SRK) is created when a user or administrator takes ownership of the system (rephrase), and works in a tree like structure to store TPM generated RSA keys used. \\ \\
The TPM offers two authorisation sessions:
\begin{itemize}
  \item Object Independent Authorisation Protocol (OIAP)
  \item Object Specific Authorisation Protocol (OSAP)
\end{itemize}
The OIAP creates a session that can manipulate any object, but will only work with certain commands. The OSAP creates a session that can only manipulate a specific object, specified at the session start. \\ \\

Note: remote attestation