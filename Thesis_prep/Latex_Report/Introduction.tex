\label{chap:Introduction}
The purpose of this report, is to establish a foundation for the future thesis by the author, in regards of extending session types to model security properties with the introduction of adversaries.

In the report I will first introduce the motivation for exploring this field, as well as the current research done in the area. Secondly, I will introduce different research areas, all related to the field, and how these work together as background knowledge for the further thesis, by which I will explain further about in the final section of this report. 

\section{Motivation}
With IT becoming an ever bigger part of our lives, the need for stronger and better security measurements, has grown with it. In recent years we have seen the introduction of voting machines in the USA and a digitalisation of our hospitals here in Denmark. With this comes the importance of being able to restrict access to ensure acceptable behaviour especially in the presence of malicious adversaries where it becomes paramount. To solve this many researchers have suggested using security protocols to ensure these security guarantees. In this report I will highlight some of the research that has made it easier to program these security protocols, as it can be a complex and error prone task. 

A lot of research has gone into communication protocols in the recent years, and fields such as Session Types, has made it a lot easier to build such protocols and ensure they are correct by construction. This idea however falls apart, when we introduce an adversary, as an attacker would be able to block messages from being delivered. This creates the motivation for doing the future thesis of introducing cryptography to Session Types, and with this report, create a basic understanding of the different fields. 

Security measurements has not only been introduced through protocols, but also through physical hardware, adding another layer of authentication. The company Trusted Computing Group introduced the Trusted Platform Module (TPM), a physical chip implemented on the motherboard, that provides a safe space for generating cryptographic keys, which is now used in a lot of modern computers and as a part of the Microsoft BitLocker. The TPM will be used as case in the thesis and is therefore also presented and explained in the following report. 

%Tighter restrictions (session types - adversaries) \\
%TPM (new security measurement) \\

\section{Current research}
The project relies heavily on the research done within the field of Security protocols and Session types. A lot of research has already gone into the field of the TPM specifications, especially by Ryan, Delaune, Kremer \textit{et al.} \autocite{DBLP:conf/ifip1-7/DelauneKRS10}, and will work as a foundation for the TPM's commands and protocols. Furthermore the report will use \citeauthor{DBLP:journals/ftpl/CortierK14,} to formally model security protocols and their goals.

\section{Intended outcome}
The intended outcome of the thesis, will be to take the idea of session types and consider them in an adversarial environment. This will be done by extending session types to model properties, and
from this produce protocols that are secure by construction. Furthermore a compiler will be constructed to automate the process of producing these programs afterwards. 